\hypertarget{bibliography}{%
\chapter*{Bibliography}\label{bibliography}}
\addcontentsline{toc}{chapter}{Bibliography}

\leavevmode\vadjust pre{\hypertarget{ref-abraham2008}{}}%
Abraham, Werner. 2008. Absent arguments on the absentive: An exercise in
silent syntax. Grammatical category or just pragmatic inference?
\emph{Language Typology and Universals} 61(4). 358--374.
doi:\href{https://doi.org/10.1524/stuf.2008.0029}{10.1524/stuf.2008.0029}.

\leavevmode\vadjust pre{\hypertarget{ref-agel2000}{}}%
Ágel, Vilmos. 2000. \emph{Valenztheorie}. Tübingen: Narr.

\leavevmode\vadjust pre{\hypertarget{ref-aichinger1754}{}}%
Aichinger, Carl Friedrich. 1754. \emph{Versuch einer teutschen
{Sprachlehre}, anfänglich nur zu eignem {Gebrauche} unternommen, endlich
aber, um den {Gelehrten} zu fernerer {Untersuchung} {Anlaß} zu geben}.
Wien: Kraus. \url{https://books.google.de/books?id=JzVGAAAAcAAJ}.

\leavevmode\vadjust pre{\hypertarget{ref-aichinger1776}{}}%
Aichinger, Carl Friedrich. 1776. Anmerkung zum zwölften {Stück} des
schwäbischen {Magazin}. \emph{Schwäbisches {Magazin} von gelehrten
{Sachen}} 627--629.
\url{https://books.google.de/books?id=bJIpAAAAYAAJ\&pg=PA627}.

\leavevmode\vadjust pre{\hypertarget{ref-bech1955}{}}%
Bech, Gunnar. 1955. \emph{Studien über das deutsche {Verbum} infinitum}.
København: Munksgaard.

\leavevmode\vadjust pre{\hypertarget{ref-becker1833}{}}%
Becker, Karl Ferdinand. 1833. \emph{Leitfaden für den ersten
{Unterricht} in der deutschen {Sprachlehre}}. Frankfurt am Main:
Hermannsche Buchhandlung.
\url{https://books.google.de/books?id=KlwSAAAAIAAJ}.

\leavevmode\vadjust pre{\hypertarget{ref-berik2015}{}}%
Berik, Olga \& Berit Gehrke. 2015. An introduction to the syntax and
semantics of pseudo-incorporation. In Olga Berik \& Berit Gehrke (eds.),
\emph{The syntax and semantics of pseudo-incorporation} (Syntax and
Semantics 40), 1--43. Leiden: Brill.
doi:\href{https://doi.org/10.1163/9789004291089_002}{10.1163/9789004291089\_002}.

\leavevmode\vadjust pre{\hypertarget{ref-booij2003b}{}}%
Booij, Geert \& Ans van Kemenade. 2003. Preverbs: An introduction. In
Geert Booij \& Jaap van Marle (eds.), \emph{Yearbook of morphology
2003}, 1--12. Dordrecht: Kluwer.
\url{https://vdoc.pub/documents/yearbook-of-morphology-2003-58l4qmnkda90}.

\leavevmode\vadjust pre{\hypertarget{ref-broschart2000}{}}%
Broschart, Jürgen. 2000. The {Tongan} category of preverbials. In Petra
M. Vogel \& Bernard Comrie (eds.), \emph{Approaches to the typology of
word classes} (Empirical Approaches to Language Typology 23), 351--370.
Berlin: De Gruyter Mouton.
doi:\href{https://doi.org/10.1515/9783110806120.351}{10.1515/9783110806120.351}.

\leavevmode\vadjust pre{\hypertarget{ref-businger2017}{}}%
Businger, Martin. 2011. \emph{{`Haben'} als {Vollverb}: Eine
dekompositionale {Analyse}} (Linguistische Arbeiten 538). Berlin: De
Gruyter.
doi:\href{https://doi.org/10.1515/9783110252644}{10.1515/9783110252644}.

\leavevmode\vadjust pre{\hypertarget{ref-carlberg1948}{}}%
Carlberg, Björn. 1948. \emph{Subjektsvertauschung und
{Objektsvertauschung} im {Deutschen}: Eine semasiologische {Studie}}.
Lund: Berlingska boktryckeriet.
\url{https://nbn-resolving.org/urn:nbn:se:su:diva-74473}.

\leavevmode\vadjust pre{\hypertarget{ref-chang2007}{}}%
Chang, Lingling. 2007. Resultative {Prädikate} und {Verbpartikeln}.
\emph{Deutsch als Fremdsprache} (2). 81--89.
doi:\href{https://doi.org/10.37307/j.2198-2430.2007.02.04}{10.37307/j.2198-2430.2007.02.04}.

\leavevmode\vadjust pre{\hypertarget{ref-colomo2010}{}}%
Colomo, Katarina. 2010. \emph{Modalität im {Verbalkomplex}:
{Halbmodalverben} und {Modalitätsverben} im {System} statusregierender
{Verbklassen}}. Ruhr-Universität Bochum PhD thesis.
\url{https://nbn-resolving.org/urn:nbn:de:hbz:294-35533}.

\leavevmode\vadjust pre{\hypertarget{ref-comrie1976}{}}%
Comrie, Bernard. 1976. \emph{Aspect: An introduction to the study of
verbal aspect and related problems}. Cambridge; New York: Cambridge
University Press.

\leavevmode\vadjust pre{\hypertarget{ref-coupe2015}{}}%
Coupé, Griet. 2015. \emph{Syntactic extension: The historical
development of {Dutch} verb clusters}. Radboud University Nijmegen PhD
thesis. \url{http://hdl.handle.net/2066/141109}.

\leavevmode\vadjust pre{\hypertarget{ref-cysouw2014a}{}}%
Cysouw, Michael. 2014. Inducing semantic roles. In Silvia Luraghi \&
Heiko Narrog (eds.), \emph{Perspectives on semantic roles}, 23--68.
Amsterdam: Benjamins.
doi:\href{https://doi.org/10.1075/tsl.106.02cys}{10.1075/tsl.106.02cys}.

\leavevmode\vadjust pre{\hypertarget{ref-czicza2014}{}}%
Czicza, Dániel. 2014. \emph{Das {es-Gesamtsystem} im {Neuhochdeutschen}}
(Studia Linguistica Germanica 120). Berlin: De Gruyter.
doi:\href{https://doi.org/10.1515/9783110357561}{10.1515/9783110357561}.

\leavevmode\vadjust pre{\hypertarget{ref-devaere2018}{}}%
De Vaere, Hilde, Ludovic De Cuypere \& Klaas Willems. 2018. Alternating
constructions with ditransitive {`geben'} in present-day {German}.
\emph{Corpus Linguistics and Linguistic Theory}.
doi:\href{https://doi.org/10.1515/cllt-2017-0072}{10.1515/cllt-2017-0072}.

\leavevmode\vadjust pre{\hypertarget{ref-diewald2010}{}}%
Diewald, Gabriele \& Elena Smirnova. 2010. \emph{Evidentiality in
{German}: Linguistic realization and regularities in grammaticalization}
(Trends in Linguistics Studies and Monographs 228). Berlin: De Gruyter
Mouton.
doi:\href{https://doi.org/10.1515/9783110241037}{10.1515/9783110241037}.

\leavevmode\vadjust pre{\hypertarget{ref-dixon1979}{}}%
Dixon, R. M. W. 1979. Ergativity. \emph{Language} 55(1). 59--138.
doi:\href{https://doi.org/10.2307/412519}{10.2307/412519}.

\leavevmode\vadjust pre{\hypertarget{ref-dixon2014}{}}%
Dixon, R. M. W. 2014. \emph{Basic linguistic theory}. Vol. 3: Further
Grammatical Topics. Oxford: Oxford University Press.

\leavevmode\vadjust pre{\hypertarget{ref-dixon2000a}{}}%
Dixon, R. M. W. \& Alexandra A. Y. Aikhenvald. 2000. Introduction. In R.
M. W. Dixon \& Alexandra A. Y. Aikhenvald (eds.), \emph{Changing
valency: Case studies in transitivity}, 1--29. Cambridge: Cambridge
University Press.
doi:\href{https://doi.org/10.1017/CBO9780511627750.002}{10.1017/CBO9780511627750.002}.

\leavevmode\vadjust pre{\hypertarget{ref-duden2009}{}}%
Duden-Grammatik. 2009. \emph{Die {Grammatik}: Unentbehrlich für
richtiges {Deutsch}} (Duden 4). 8. überarbeitete {Auflage}. Mannheim:
Dudenverlag.

\leavevmode\vadjust pre{\hypertarget{ref-dux2020}{}}%
Dux, Ryan. 2020. \emph{Frame-constructional verb classes: Change and
theft verbs in {English} and {German}} (Constructional Approaches to
Language 28). Amsterdam: Benjamins.
doi:\href{https://doi.org/10.1075/cal.28}{10.1075/cal.28}.

\leavevmode\vadjust pre{\hypertarget{ref-eisenberg2006}{}}%
Eisenberg, Peter. 2006b. \emph{Grundriss der deutschen {Grammatik} 1:
{Das} {Wort}}. 3rd edition. Stuttgart: Metzler.

\leavevmode\vadjust pre{\hypertarget{ref-eisenberg2006a}{}}%
Eisenberg, Peter. 2006a. \emph{Grundriss der deutschen {Grammatik} 2:
{Der Satz}}. 3rd edition. Stuttgart: Metzler.

\leavevmode\vadjust pre{\hypertarget{ref-engel1996}{}}%
Engel, Ulrich. 1996. \emph{Deutsche {Grammatik}}. (3. korrigierte
Auflage). Heidelberg: Julius Groos.

\leavevmode\vadjust pre{\hypertarget{ref-engel1978}{}}%
Engel, Ulrich \& Helmut Schumacher. 1978. \emph{Kleines {Valenzlexikon}
deutscher {Verben}} ({Forschungsberichte des Instituts für deutsche
Sprache} 31). Tübingen: Narr.

\leavevmode\vadjust pre{\hypertarget{ref-engelen1986}{}}%
Engelen, Bernhard. 1986. \emph{Einführung in die {Syntax} der deutschen
{Sprache}}. Vol. 2. Balemannsweiler: Pädagogische Verlag Burgbücherei
Schneider.

\leavevmode\vadjust pre{\hypertarget{ref-enzinger2012}{}}%
Enzinger, Stefan. 2012. \emph{Kausative und perzeptive
{Infinitivkonstruktionen}} (Studia Grammatica 70). Berlin: Akademie
Verlag.
doi:\href{https://doi.org/10.1524/9783050062310}{10.1524/9783050062310}.

\leavevmode\vadjust pre{\hypertarget{ref-erb2001}{}}%
Erb, Marie Christine. 2001. \emph{Finite auxiliaries in {German}}.
Tilburg: Katholieke Universiteit Brabant PhD thesis.
\url{https://research.tilburguniversity.edu/en/publications/finite-auxiliaries-in-german}.

\leavevmode\vadjust pre{\hypertarget{ref-eroms1980}{}}%
Eroms, Hans-Werner. 1980. \emph{\emph{Be}-{Verb} und
{Präpositionalphrase}: Ein {Beitrag} zur {Grammatik} der deutschen
{Verbalpräfixe}} (Monographien {zur} Sprachwissenschaft 9). Heidelberg:
Winter.

\leavevmode\vadjust pre{\hypertarget{ref-eroms2000}{}}%
Eroms, Hans-Werner. 2000. \emph{Syntax der deutschen {Sprache}} (De
Gruyter Studienbuch). Berlin: De Gruyter.
doi:\href{https://doi.org/10.1515/9783110808124}{10.1515/9783110808124}.

\leavevmode\vadjust pre{\hypertarget{ref-fabriciushansen1986}{}}%
Fabricius-Hansen, Christine. 1986. \emph{Tempus fugit: Über die
{Interpretation} temporaler {Strukturen} im {Deutschen}} (Sprache {der}
Gegenwart 64). Düsseldorf: Schwann.

\leavevmode\vadjust pre{\hypertarget{ref-fehrmann2018}{}}%
Fehrmann, Ingo. 2018. \emph{Kausative {Konstruktionen} mit dem {Verb}
'machen' im {Deutschen}}. Berlin: Humboldt-Universität PhD thesis.
doi:\href{https://doi.org/10.18452/19403}{10.18452/19403}.

\leavevmode\vadjust pre{\hypertarget{ref-felfe2012}{}}%
Felfe, Marc. 2012. \emph{Das {System} der {Partikelverben} mit 'an':
Eine konstruktionsgrammatische {Untersuchung}} (Sprache {und} Wissen
12). Berlin: De Gruyter.
doi:\href{https://doi.org/10.1515/9783110289930}{10.1515/9783110289930}.

\leavevmode\vadjust pre{\hypertarget{ref-fischer2020}{}}%
Fischer, Hanna. 2020. \emph{Präteritumschwund im {Deutschen}:
{Dokumentation} und {Erklärung} eines {Verdrängungsprozesses}} (Studia
Linguistica Germanica 132). Berlin: De Gruyter.
doi:\href{https://doi.org/10.1515/9783110563818}{10.1515/9783110563818}.

\leavevmode\vadjust pre{\hypertarget{ref-fleischhauer2018}{}}%
Fleischhauer, Jens. 2018. Animacy and affectedness in {Germanic}
languages. \emph{Open Linguistics} 4(1). 566--588.
doi:\href{https://doi.org/10.1515/opli-2018-0028}{10.1515/opli-2018-0028}.

\leavevmode\vadjust pre{\hypertarget{ref-fleischhauer2021}{}}%
Fleischhauer, Jens. 2021. Light verb constructions and their families: A
corpus study on {G}erman \emph{stehen unter}-{LVC}s. In Paul Cook,
Jelena Mitrović, Carla Parra Escartín, Ashwini Vaidya, Petya Osenova,
Shiva Taslimipoor \& Carlos Ramisch (eds.), \emph{Proceedings of the
17th workshop on multiword expressions (MWE 2021)}, 63--69. Association
for Computational Linguistics.
doi:\href{https://doi.org/10.18653/v1/2021.mwe-1.8}{10.18653/v1/2021.mwe-1.8}.

\leavevmode\vadjust pre{\hypertarget{ref-fleischhauer2019}{}}%
Fleischhauer, Jens \& Thomas Gamerschlag. 2019. Deriving the meaning of
light verb constructions: A frame account of {German} \emph{stehen}
{`stand'}. \emph{Yearbook of the German Cognitive Linguistics
Association} 7(1). 137--156.
doi:\href{https://doi.org/10.1515/gcla-2019-0009}{10.1515/gcla-2019-0009}.

\leavevmode\vadjust pre{\hypertarget{ref-fleischhauer2021a}{}}%
Fleischhauer, Jens \& Stefan Hartmann. 2021. The emergence of light verb
constructions: A case study on {German} \emph{kommen} {`come'}.
\emph{Yearbook of the German Cognitive Linguistics Association} 9(1).
135--156.
doi:\href{https://doi.org/10.1515/gcla-2021-0007}{10.1515/gcla-2021-0007}.

\leavevmode\vadjust pre{\hypertarget{ref-fuhrhop2012}{}}%
Fuhrhop, Nanna. 2012. \emph{Zwischen {Wort} und {Syntagma}: Zur
grammatischen {Fundierung} der {Getrennt-} und {Zusammenschreibung}}
(Linguistische Arbeiten 513). Tübingen: Niemeyer.
doi:\href{https://doi.org/10.1515/9783110936544}{10.1515/9783110936544}.

\leavevmode\vadjust pre{\hypertarget{ref-fuss2017}{}}%
Fuß, Eric, Marek Konopka \& Angelika Wöllstein. 2017. Perspektiven auf
syntaktische {Variation}. In Marek Konopka \& Angelika Wöllstein (eds.),
\emph{Grammatische {Variation}: {Empirische Zugänge} und theoretische
{Modellierung}}, 229--254. Berlin: De Gruyter.
doi:\href{https://doi.org/10.1515/9783110518214-014}{10.1515/9783110518214-014}.

\leavevmode\vadjust pre{\hypertarget{ref-gallmann1999}{}}%
Gallmann, Peter. 1999. Wortbegriff und {Nomen‐Verb‐Verbindungen}.
\emph{Zeitschrift für Sprachwissenschaft} 18(2). 269--304.
doi:\href{https://doi.org/10.1515/zfsw.1999.18.2.269}{10.1515/zfsw.1999.18.2.269}.

\leavevmode\vadjust pre{\hypertarget{ref-gamerschlag2014}{}}%
Gamerschlag, Thomas. 2014. Stative dimensional verbs in {German}.
\emph{Studies in Language} 38(2). 275--334.
doi:\href{https://doi.org/10.1075/sl.38.2.02gam}{10.1075/sl.38.2.02gam}.

\leavevmode\vadjust pre{\hypertarget{ref-gargyan2010}{}}%
Gárgyán, Gabriella. 2010. \emph{Der \emph{am}-{Progressiv} im heutigen
{Deutsch}}. Szeged: University of Szeged PhD thesis.
doi:\href{https://doi.org/10.14232/phd.788}{10.14232/phd.788}.

\leavevmode\vadjust pre{\hypertarget{ref-geist2016}{}}%
Geist, Ljudmila \& Daniel Hole. 2016. Theta-head binding in the {German}
locative alternation. In Nadine Bade, Polina Berezovskaya \& Anthea
Schöller (eds.), \emph{Proceedings of {Sinn} und {Bedeutung} 20},
270--287. University of Tübingen.
doi:\href{https://doi.org/10.18148/sub/2016.v20i0.263}{10.18148/sub/2016.v20i0.263}.

\leavevmode\vadjust pre{\hypertarget{ref-geniusiene1987}{}}%
Geniušė, Emma. 1987. \emph{The typology of reflexives} (Empirical
Approaches to Language Typology 2). Berlin: Mouton de Gruyter.
doi:\href{https://doi.org/10.1515/9783110859119}{10.1515/9783110859119}.

\leavevmode\vadjust pre{\hypertarget{ref-gillmann2016}{}}%
Gillmann, Melitta. 2016. \emph{Perfektkonstruktionen mit {`haben'} und
{`sein'}} (Studia Linguistica Germanica 128). Berlin: De Gruyter.
doi:\href{https://doi.org/10.1515/9783110492170}{10.1515/9783110492170}.

\leavevmode\vadjust pre{\hypertarget{ref-goldberg2006}{}}%
Goldberg, Adele E. 2006. \emph{Constructions at work: The nature of
generalization in language}. Oxford: Oxford University Press.

\leavevmode\vadjust pre{\hypertarget{ref-grewendorf1989}{}}%
Grewendorf, Günther. 1989. \emph{Ergativity in {German}} (Studies in
Generative Grammar 35). Dordrecht: Foris.
doi:\href{https://doi.org/10.1515/9783110859256}{10.1515/9783110859256}.

\leavevmode\vadjust pre{\hypertarget{ref-groot2000}{}}%
Groot, Casper de. 2000. The absentive. In Östen Dahl (ed.), \emph{Tense
and aspect in the languages of {Europe}} (Empirical Approaches to
Language Typology 20-6), 693--722. Berlin: De Gruyter Mouton.
doi:\href{https://doi.org/10.1515/9783110197099.4.693}{10.1515/9783110197099.4.693}.

\leavevmode\vadjust pre{\hypertarget{ref-gunther1987}{}}%
Günther, Hartmut. 1987. Wortbildung, {Syntax}, \emph{be}-{Verben} und
das {Lexikon}. \emph{Beiträge zur Geschichte der deutschen Sprache und
Literatur} 109. 179--201.
doi:\href{https://doi.org/10.1515/bgsl.1987.1987.109.179}{10.1515/bgsl.1987.1987.109.179}.

\leavevmode\vadjust pre{\hypertarget{ref-haider2010}{}}%
Haider, Hubert. 2010. \emph{The syntax of {German}} (Cambridge Syntax
Guides). Cambridge: Cambridge University Press.
doi:\href{https://doi.org/10.1017/CBO9780511845314}{10.1017/CBO9780511845314}.

\leavevmode\vadjust pre{\hypertarget{ref-haig2005}{}}%
Haig, Geoffrey. 2005. Bescheuert und verlogen: {(Schein)partizipien},
{Wortklassen}, und das {Lexikon}. In Yvonne Thiesen (ed.), \emph{{10
Jahre Ulrike Mosel am SAVS: Beiträge ihrer Absolventen zum
Dienstjubiläum}} (SAVS Arbeitsberichte 4), 107--128. Kiel: Seminar für
Allgemeine und Vergleichende Sprachwissenschaft.
\url{https://www.isfas.uni-kiel.de/de/linguistik/forschung/uploads/arbeitsberichte-uploads/alte-berichte/10-jahre-mosel}.

\leavevmode\vadjust pre{\hypertarget{ref-harbert1977}{}}%
Harbert, Wayne. 1977. Clause union and {German} accusative plus
infinitive constructions. In Peter Cole \& Jerrold M. Sadock (eds.),
\emph{Grammatical relations} (Syntax and Semantics 8), 121--150. New
York: Academic Press.

\leavevmode\vadjust pre{\hypertarget{ref-harris1957}{}}%
Harris, Zellig S. 1957. Co-occurrence and transformation in linguistic
structure ({Presidential} address to the {Linguistic Society of America}
1955). \emph{Language} 33(3). 283--340.
doi:\href{https://doi.org/10.2307/411155}{10.2307/411155}.

\leavevmode\vadjust pre{\hypertarget{ref-hartmann2014}{}}%
Hartmann, Iren, Martin Haspelmath \& Michael Cysouw. 2014. Identifying
semantic role clusters and alignment types via microrole coexpression
tendencies. \emph{Studies in Language} 38(3). 463--484.
doi:\href{https://doi.org/10.1075/sl.38.3.02har}{10.1075/sl.38.3.02har}.

\leavevmode\vadjust pre{\hypertarget{ref-haspelmath1987}{}}%
Haspelmath, Martin. 1987. \emph{Transitivity alternations of the
anticausative type} (Arbeitspapier, Neue Folge 5). Köln: Institut für
Linguistik. \url{http://nbn-resolving.org/urn:nbn:de:hebis:30:3-243207}.

\leavevmode\vadjust pre{\hypertarget{ref-haspelmath1989}{}}%
Haspelmath, Martin. 1989. From purposive to infinitive: A universal path
of grammaticization. \emph{Folia Linguistica Historica} 10(1/2).
287--310.
doi:\href{https://doi.org/10.1515/flih.1989.10.1-2.287}{10.1515/flih.1989.10.1-2.287}.

\leavevmode\vadjust pre{\hypertarget{ref-haspelmath1993a}{}}%
Haspelmath, Martin. 1993. More on the typology of inchoative/causative
verb alternations. In Bernard Comrie \& Maria Polinsky (eds.),
\emph{Causatives and transitivity} (Studies in Language Companion
Series), 87--120. Amsterdam: Benjamins.
doi:\href{https://doi.org/10.1075/slcs.23.05has}{10.1075/slcs.23.05has}.

\leavevmode\vadjust pre{\hypertarget{ref-haspelmath2005d}{}}%
Haspelmath, Martin. 2005. Argument marking in ditransitive alignment
types. \emph{Linguistic Discovery} 3(1). 1--21.
doi:\href{https://doi.org/10.1349/PS1.1537-0852.A.280}{10.1349/PS1.1537-0852.A.280}.

\leavevmode\vadjust pre{\hypertarget{ref-haspelmath2013a}{}}%
Haspelmath, Martin \& Luisa Baumann. 2013. German valency patterns. In
Iren Hartmann, Martin Haspelmath \& Bradley Taylor (eds.), \emph{Valency
patterns}. Leipzig: Max Planck Institute for Evolutionary Anthropology.
\url{http://valpal.info/languages/german}.

\leavevmode\vadjust pre{\hypertarget{ref-haspelmath2004c}{}}%
Haspelmath, Martin \& Thomas Müller-Bardey. 2004. Valency change. In
Geert Booij, Christian Lehmann, Joachim Mugdan \& Stavros Skopeteas
(eds.), \emph{Morphology} (HSK 17/2), 1130--1145. Berlin: De Gruyter
Mouton.
doi:\href{https://doi.org/10.1515/9783110172782.2.14.1130}{10.1515/9783110172782.2.14.1130}.

\leavevmode\vadjust pre{\hypertarget{ref-heine2020}{}}%
Heine, Antje. 2020. Zwischen {Grammatik} und {Lexikon}. Ein
forschungsgeschichtlicher {Blick} auf {Funktionsverbgefüge}. In Sabine
Knop \& Manon Hermann (eds.), \emph{Theoretische, didaktische und
kontrastive {Perspektiven}}, 15--38. Berlin: De Gruyter.
doi:\href{https://doi.org/10.1515/9783110697353-002}{10.1515/9783110697353-002}.

\leavevmode\vadjust pre{\hypertarget{ref-helbig1978}{}}%
Helbig, Gerhard. 1978. Zu den zustandsbezeichnenden {Konstruktionen} mit
\emph{sein} und \emph{haben} im {Deutschen}. \emph{Linguistische
Arbeitsberichte} 20. 37--46.

\leavevmode\vadjust pre{\hypertarget{ref-helbig2001}{}}%
Helbig, Gerhard \& Joachim Buscha. 2001. \emph{Deutsche {Grammatik}.
{Ein} {Handbuch} für den {Ausländerunterricht}}. München: Langenscheidt.

\leavevmode\vadjust pre{\hypertarget{ref-helbig1991}{}}%
Helbig, Gerhard \& Wolfgang Schenkel. 1969. \emph{Wörterbuch zur
{Valenz} und {Distribution} deutscher {Verben}}. (7th edition 1983,
prepared in digital form by De Gruyter 1991). Tübingen: Niemeyer.
doi:\href{https://doi.org/10.1515/9783111561486}{10.1515/9783111561486}.

\leavevmode\vadjust pre{\hypertarget{ref-heringer1968}{}}%
Heringer, Hans-Jürgen. 1968. Präpositionale {Ergänzungsbestimmungen} im
{Deutschen}. \emph{Zeitschrift für Deutsche Philologie} 87. 426--457.

\leavevmode\vadjust pre{\hypertarget{ref-himmelmann2005b}{}}%
Himmelmann, Nikolaus P. \& Eva Schulze-Berndt. 2005. Issues in the
syntax and semantics of participant-oriented adjuncts: An introduction.
In Nikolaus P. Himmelmann \& Eva Schulze-Berndt (eds.), \emph{Secondary
predication and adverbial modification: The typology of depictives},
1--67. Oxford: Oxford University Press.

\leavevmode\vadjust pre{\hypertarget{ref-hinze2007}{}}%
Hinze, Christian \& Klaus-Michael Köpcke. 2007. Was wissen
{Grundschüler} über die {Verwendung} der {Perfektauxiliare} \emph{haben}
und \emph{sein}? In Klaus-Michael Köpcke \& Arne Ziegler (eds.),
\emph{Grammatik in der {Universität} und für die {Schule}}
(Germanistische Linguistik 277), 95--128. Niemeyer.
doi:\href{https://doi.org/10.1515/9783110975918}{10.1515/9783110975918}.

\leavevmode\vadjust pre{\hypertarget{ref-hohle1978}{}}%
Höhle, Tilman N. 1978. \emph{Lexikalistische {Syntax}: Die
{Aktiv-Passiv-Relation} und andere {Infinitkonstruktionen} im
{Deutschen}} (Linguistische Arbeiten 67). Tübingen: Niemeyer.
doi:\href{https://doi.org/10.1515/9783111345444}{10.1515/9783111345444}.

\leavevmode\vadjust pre{\hypertarget{ref-hole2002}{}}%
Hole, Daniel. 2002. Er hat den {Arm} verbunden: {Valenzreduktion} und
{Argumentvermehrung} im {Haben}-{Konfigurativ}. In Mitsunobu Yoshida
(ed.), \emph{Grammatische {Kategorien} aus sprachhistorischer und
typologischer {Perspektive}. {Akten} des 29. {Linguisten-Seminars (Kyoto
2001)}}, 167--186. München: Iudicium.

\leavevmode\vadjust pre{\hypertarget{ref-hole2014}{}}%
Hole, Daniel. 2014. \emph{Dativ, {Bindung} und {Diathese}} (Studia
Grammatica 78). Berlin: De Gruyter.
doi:\href{https://doi.org/10.1515/9783110347739}{10.1515/9783110347739}.

\leavevmode\vadjust pre{\hypertarget{ref-holl2010}{}}%
Holl, Daniel. 2010. \emph{Modale {Infinitive} und dispositionelle
{Modalität} im {Deutschen}} (Studia Grammatica 71). Berlin: De Gruyter.
doi:\href{https://doi.org/10.1524/9783050062341}{10.1524/9783050062341}.

\leavevmode\vadjust pre{\hypertarget{ref-hooste2018}{}}%
Hooste, Koen Van. 2018. \emph{Instruments and related concepts at the
syntax-semantics interface} (Dissertations in Language and Cognition 5).
Düsseldorf University Press.
doi:\href{https://doi.org/10.1515/9783110720365}{10.1515/9783110720365}.

\leavevmode\vadjust pre{\hypertarget{ref-hopper1980}{}}%
Hopper, Paul J. \& Sandra A. Thompson. 1980. Transitivity in grammar and
discourse. \emph{Language} 56(1). 251--299.
doi:\href{https://doi.org/10.2307/413757}{10.2307/413757}.

\leavevmode\vadjust pre{\hypertarget{ref-hundsnurscher1968}{}}%
Hundsnurscher, Franz. 1968. \emph{Das {System} der {Partikelverben} mit
{`aus'} in der {Gegenwartssprache}} (Göppinger Arbeiten {zur}
Germanistik 2). Göppingen: Kümmerle.

\leavevmode\vadjust pre{\hypertarget{ref-imo2018}{}}%
Imo, Wolfgang. 2018. Valence patterns, construction, and interaction:
Constructs with the {German} verb \emph{erinnern} ({`remember/remind'}).
In Hans C. Boas \& Alexander Ziem (eds.), \emph{Constructional
approaches to syntactic structures in {German}} (Trends in Linguistics
Studies and Monographs 322), 131--178. Berlin: De Gruyter.
doi:\href{https://doi.org/10.1515/9783110457155-004}{10.1515/9783110457155-004}.

\leavevmode\vadjust pre{\hypertarget{ref-jager2018}{}}%
Jäger, Agnes. 2018. On the history of the {IPP} construction in
{German}. In Agnes Jäger, Gisella Ferraresi \& Helmut Weiß (eds.),
\emph{Clause structure and word order in the history of {German}}
(Oxford Studies in Diachronic and Historical Linguistics 28), 302--323.
Oxford: Oxford University Press.
doi:\href{https://doi.org/10.1093/oso/9780198813545.003.0016}{10.1093/oso/9780198813545.003.0016}.

\leavevmode\vadjust pre{\hypertarget{ref-jager2013}{}}%
Jäger, Anne. 2013. \emph{Der {Status} von \emph{bekommen + zu +}
{Infinitiv} zwischen {Modalität} und semantischer {Perspektivierung}}
(Theorie {und} Vermittlung {der} Sprache 56). Frankfurt am Main: Peter
Lang.
doi:\href{https://doi.org/10.3726/978-3-653-03239-0}{10.3726/978-3-653-03239-0}.

\leavevmode\vadjust pre{\hypertarget{ref-janic2010}{}}%
Janic, Katarzyna. 2010. On the reflexive-antipassive polysemy:
Typological convergence from unrelated languages. \emph{Annual Meeting
of the Berkeley Linguistics Society} 36(1). 158--173.
doi:\href{https://doi.org/10.3765/bls.v36i1.3909}{10.3765/bls.v36i1.3909}.

\leavevmode\vadjust pre{\hypertarget{ref-kamber2008}{}}%
Kamber, Alain. 2008. \emph{Funktionsgefüge - empirisch: Eine
korpusbasierte {Untersuchung} zu den nominalen {Prädikaten} des
{Deutschen}} (Germanistische Linguistik 281). Tübingen: Niemeyer.
doi:\href{https://doi.org/10.1515/9783484970311}{10.1515/9783484970311}.

\leavevmode\vadjust pre{\hypertarget{ref-keller2003}{}}%
Keller, Frank \& Antonella Sorace. 2003. Gradient auxiliary selection
and impersonal passivization in {German}: An experimental investigation.
\emph{Journal of Linguistics} 39(1). 57--108.
\url{http://www.jstor.org/stable/4176789}.

\leavevmode\vadjust pre{\hypertarget{ref-kim1983}{}}%
Kim, Gyung-Uk. 1983. \emph{Valenz und {Wortbildung}: Dargestellt am
{Beispiel} der verbalen {Präfixbildung} mit be-, ent-, er-, miss-, ver-,
zer-}. Würzburg: Königshausen + Neumann.

\leavevmode\vadjust pre{\hypertarget{ref-kiss1995}{}}%
Kiss, Tibor. 1995. \emph{Infinite {Komplementation}: Neue {Studien} zum
deutschen {Verbum} {Infinitum}} (Linguistische Arbeiten 333). Tübingen:
Niemeyer.
doi:\href{https://doi.org/10.1515/9783110934670}{10.1515/9783110934670}.

\leavevmode\vadjust pre{\hypertarget{ref-konig2009}{}}%
König, Svenja. 2009. \emph{Alle sind {Deutschland} ... {außer} {Fritz
Eckenga} -- der ist einkaufen!} {Der Absentiv} in der deutschen
{Gegenwartssprache}. In Edeltraud Winkler (ed.), \emph{Konstruktionelle
{Varianz} bei {Verben}} (Online {publizierte} Arbeiten {zur} Linguistik
4/2009), 42--74. Mannheim: Institut für Deutsche Sprache.
\url{https://nbn-resolving.org/urn:nbn:de:bsz:mh39-349}.

\leavevmode\vadjust pre{\hypertarget{ref-konopka2021}{}}%
Konopka, Marek \& Sandra Hansen-Morath. 2021. {AcI-Konstruktionen}.
\emph{Korpusgrammatik}. Mannheim: Leibniz-Institut für Deutsche Sprache.
doi:\href{https://doi.org/10.14618/korpusgrammatik}{10.14618/korpusgrammatik}.

\leavevmode\vadjust pre{\hypertarget{ref-kotulkova2010a}{}}%
Kotůlková, Veronika. 2010a. \emph{Die {Vielfalt} der
{lassen+Infinitiv-Konstruktion} im {Deutschen} und wie das
{Tschechische} damit zurechtkommt} (DeuCze: Korpuslinguistik
Deutsch-Tschechisch Kontrastiv 1). Würzburg: Universität Würzburg.
doi:\href{https://doi.org/10.25972/OPUS-4217}{10.25972/OPUS-4217}.

\leavevmode\vadjust pre{\hypertarget{ref-kotulkova2010}{}}%
Kotůlková, Veronika. 2010b. Kontrastive {Bemerkungen} zu
{Konstruktionen} mit {Wahrnehmungsverben}. \emph{Brünner Beiträge zur
Germanistik und Nordistik} 15(1-2). 21--35.
\url{http://hdl.handle.net/11222.digilib/114735}.

\leavevmode\vadjust pre{\hypertarget{ref-kramer2004}{}}%
Krämer, Sabine. 2004. \emph{Bleiben} bleibt \emph{bleiben}.
\emph{Zeitschrift für Sprachwissenschaft} 23(2). 245--274.
doi:\href{https://doi.org/10.1515/zfsw.2004.23.2.245}{10.1515/zfsw.2004.23.2.245}.

\leavevmode\vadjust pre{\hypertarget{ref-krause2002}{}}%
Krause, Olaf. 2002. \emph{Progressiv im {Deutschen}: Eine empirische
{Untersuchung} im {Kontrast} mit {Niederländisch} und {Englisch}}
(Linguistische Arbeiten 462). Tübingen: Niemeyer.
doi:\href{https://doi.org/10.1515/9783110916454}{10.1515/9783110916454}.

\leavevmode\vadjust pre{\hypertarget{ref-kruisinga1935}{}}%
Kruisinga, Etsko. 1935. \emph{Einführung in die deutsche {Syntax}}.
Groningen: Noordhoff.

\leavevmode\vadjust pre{\hypertarget{ref-kubczak2016}{}}%
Kubczak, Jacqueline. 2014. \emph{Er kann {Kanzler}! Wir können billig!}:
Schwer zu fassende {Neuerungen} in der deutschen {Sprache}! \emph{AION.
Annali di Università degli Studi di Napoli L'Orientale, Sezione
Germanica.} 24(1-2). 127--139.
\url{https://nbn-resolving.org/urn:nbn:de:bsz:mh39-47260}.

\leavevmode\vadjust pre{\hypertarget{ref-kulikov2011}{}}%
Kulikov, Leonid. 2011. Voice typology. In Jae Jung Song (ed.), \emph{The
{Oxford} handbook of linguistic typology}, 368--398. Oxford: Oxford
University Press.
doi:\href{https://doi.org/10.1093/oxfordhb/9780199281251.013.0019}{10.1093/oxfordhb/9780199281251.013.0019}.

\leavevmode\vadjust pre{\hypertarget{ref-kunze1996}{}}%
Kunze, Jürgen. 1996. Plain middles and \emph{lassen} middles in
{German}: Reflexive constructions and sentence perspective.
\emph{Linguistics} 34(3). 645--95.
doi:\href{https://doi.org/10.1515/ling.1996.34.3.645}{10.1515/ling.1996.34.3.645}.

\leavevmode\vadjust pre{\hypertarget{ref-kunze1997}{}}%
Kunze, Jürgen. 1997. Typen der reflexiven {Verbverwendung} im
{Deutschen} und ihre {Herkunft}. \emph{Zeitschrift für
Sprachwissenschaft} 16(1/2). 83--180.
doi:\href{https://doi.org/10.1515/zfsw.1997.16.1-2.83}{10.1515/zfsw.1997.16.1-2.83}.

\leavevmode\vadjust pre{\hypertarget{ref-kurogo2016}{}}%
Kurogo, Yoko. 2016. Aspektuelle {Interpretation} von antikausativen
{Verben} im {Deutschen}. \emph{Dokkyo Universität Germanistische
Forschungsbeiträge} (71). 25--40.
\url{http://id.nii.ac.jp/1140/00000972/}.

\leavevmode\vadjust pre{\hypertarget{ref-lasch2016}{}}%
Lasch, Alexander. 2016. \emph{Nonagentive {Konstruktionen} des
{Deutschen}} (Sprache {und} Wissen 25). Berlin: De Gruyter.
doi:\href{https://doi.org/10.1515/9783110495430}{10.1515/9783110495430}.

\leavevmode\vadjust pre{\hypertarget{ref-lasch2018}{}}%
Lasch, Alexander. 2018. \emph{Diese gehören kalt zu geben}: Die
{Konstruktion} \emph{gehören} mit {Qualitativ}.
\emph{Sprachwissenschaft} 43(2). 159--185.
\url{https://sprw.winter-verlag.de/article/SPRW/2018/2/5}.

\leavevmode\vadjust pre{\hypertarget{ref-latzel1977a}{}}%
Latzel, Sigbert. 1977a. \emph{Haben} + {Partizip} und ähnliche
{Verbindungen}. \emph{Deutsche Sprache} 5(4). 289--312.

\leavevmode\vadjust pre{\hypertarget{ref-latzel1977}{}}%
Latzel, Sigbert. 1977b. \emph{Die deutschen {Tempora} {Perfekt} und
{Präteritum}: Eine {Darstellung} mit {Bezug} auf {Erfordernisse} des
{Faches} {`{Deutsch} als {Fremdsprache}'}} (Heutiges Deutsch Reihe III,
Band 2). München: Hueber.

\leavevmode\vadjust pre{\hypertarget{ref-leirbukt1981}{}}%
Leirbukt, Oddleif. 1981. {`Passivähnliche'} {Konstruktionen} mit
\emph{haben} + {Partizip} im heutigen {Deutsch}. \emph{Deutsche Sprache}
9. 119--146.

\leavevmode\vadjust pre{\hypertarget{ref-leirbukt1997}{}}%
Leirbukt, Oddleif. 1997. \emph{Untersuchungen zum
\emph{bekommen}-{Passiv} im heutigen {Deutsch}} (Germanistische
Linguistik 177). Tübingen: Niemeyer.
doi:\href{https://doi.org/10.1515/9783110928013}{10.1515/9783110928013}.

\leavevmode\vadjust pre{\hypertarget{ref-leirbukt2000}{}}%
Leirbukt, Oddleif. 2000. Passivähnliche {Bildungen} mit
\emph{haben/wissen/sehen} + {Partizip II} in modalen {Kontexten}. In
Rolf Thieroff, Matthias Tamrat, Nanna Fuhrhop \& Oliver Teuber (eds.),
\emph{{Deutsche Grammatik in Theorie und Praxis}}, 97--110. Tübingen:
Niemeyer.
doi:\href{https://doi.org/10.1515/9783110933932.97}{10.1515/9783110933932.97}.

\leavevmode\vadjust pre{\hypertarget{ref-lenz2013}{}}%
Lenz, Alexandra N. 2013. \emph{Vom \emph{kriegen} und \emph{bekommen}}
(Linguistik: Impulse \& Tendenzen 53). Berlin: De Gruyter.
doi:\href{https://doi.org/10.1515/9783110314915}{10.1515/9783110314915}.

\leavevmode\vadjust pre{\hypertarget{ref-levin1993}{}}%
Levin, Beth. 1993. \emph{English verb classes and alternations: A
preliminary investigation}. Chicago: University of Chicago Press.

\leavevmode\vadjust pre{\hypertarget{ref-lipka1972}{}}%
Lipka, Leonhard. 1972. \emph{Semantic structure and word-formation:
Verb-particle constructions in contemporary {English}} (International
Library of General Linguistics 17). München: Fink.
\url{https://nbn-resolving.org/urn:nbn:de:bvb:19-epub-5050-4}.

\leavevmode\vadjust pre{\hypertarget{ref-los2005}{}}%
Los, Bettelou. 2005. \emph{The rise of the \emph{to}-infinitive}.
Oxford: Oxford University Press.
doi:\href{https://doi.org/10.1093/acprof:oso/9780199274765.001.0001}{10.1093/acprof:oso/9780199274765.001.0001}.

\leavevmode\vadjust pre{\hypertarget{ref-los2016}{}}%
Los, Bettelou, Corrien Blom, Geert Booij, Marion Elenbaas \& Ans van
Kemenade. 2016. \emph{Morphosyntactic change: A comparative study of
particles and prefixes} (Cambridge Studies in Linguistics 134).
Cambridge: Cambridge University Press.
doi:\href{https://doi.org/10.1017/CBO9780511998447}{10.1017/CBO9780511998447}.

\leavevmode\vadjust pre{\hypertarget{ref-maienborn2007}{}}%
Maienborn, Claudia. 2008. Das {Zustandspassiv}: Grammatische
{Einordnung}--{Bildungsbeschränkung}--{Interpretationsspielraum}.
\emph{Zeitschrift für germanistische Linguistik} 35(1-2). 83--114.
doi:\href{https://doi.org/10.1515/ZGL.2007.005}{10.1515/ZGL.2007.005}.

\leavevmode\vadjust pre{\hypertarget{ref-malchukov2015}{}}%
Malchukov, Andrej. 2015. Valency classes and alternations: Parameters of
variation. In Andrej Malchukov \& Bernard Comrie (eds.), \emph{Valency
classes in the world's languages} (Comparative Handbooks of
Linguistics), vol. 1, 73--130. Berlin: De Gruyter Mouton.
doi:\href{https://doi.org/10.1515/9783110338812-007}{10.1515/9783110338812-007}.

\leavevmode\vadjust pre{\hypertarget{ref-malchukov2015a}{}}%
Malchukov, Andrej \& Bernard Comrie (eds.). 2015. \emph{Valency classes
in the world's languages}. Berlin: De Gruyter Mouton.
doi:\href{https://doi.org/10.1515/9783110338812}{10.1515/9783110338812}.

\leavevmode\vadjust pre{\hypertarget{ref-marcotte2010}{}}%
Marcotte, Ethan. 2010. Responsive web design. \emph{A List Apart} 306.
\url{https://alistapart.com/article/responsive-web-design/}.

\leavevmode\vadjust pre{\hypertarget{ref-mcintyre2001}{}}%
McIntyre, Andrew. 2001. \emph{German double particles as preverbs:
Morphology and conceptual semantics} (Studien {zur} {deutschen}
Grammatik 61). Tübingen: Stauffenburg.

\leavevmode\vadjust pre{\hypertarget{ref-mcintyre2003}{}}%
McIntyre, Andrew. 2003. Preverbs, argument linking and verb semantics:
{Germanic} prefixes and particles. In Geert Booij \& Jaap van Marle
(eds.), \emph{Yearbook of morphology 2003}, 119--144. Dordrecht: Kluwer.
doi:\href{https://doi.org/10.1007/978-1-4020-1513-7_6}{10.1007/978-1-4020-1513-7\_6}.

\leavevmode\vadjust pre{\hypertarget{ref-melcuk1993}{}}%
Mel'čuk, Igor. 1993. The inflectional category of voice: Towards a more
rigorous definition. In Bernard Comrie \& Maria Polinsky (eds.),
\emph{Causatives and transitivity} (Studies in Language Companion Series
23), 1--46. Amsterdam: Benjamins.
doi:\href{https://doi.org/10.1075/slcs.23.02mel}{10.1075/slcs.23.02mel}.

\leavevmode\vadjust pre{\hypertarget{ref-mithun1991}{}}%
Mithun, Marianne. 1991. Active/agentive case marking and its
motivations. \emph{Language} 67(3). 510--546.
doi:\href{https://doi.org/10.2307/415036}{10.2307/415036}.

\leavevmode\vadjust pre{\hypertarget{ref-nedjalkov1976}{}}%
Nedjalkov, Vladimir P. 1976. \emph{Kausativkonstruktionen} (Studien
{zur} {deutschen} Grammatik 4). Tübingen: Narr.

\leavevmode\vadjust pre{\hypertarget{ref-nedjalkov1988a}{}}%
Nedjalkov, Vladimir P. 1988. Resultative, passive, and perfect in
{German}. In Vladimir P. NedjaÌlkov (ed.), \emph{Typology of resultative
constructions} (Typological Studies in Language 12), 411--432.
Amsterdam: Benjamins.
doi:\href{https://doi.org/10.1075/tsl.12.29ned}{10.1075/tsl.12.29ned}.

\leavevmode\vadjust pre{\hypertarget{ref-nichols2004}{}}%
Nichols, Johanna, David A. Peterson \& Jonathan Barnes. 2004.
Transitivizing and detransitivizing languages. \emph{Linguistic
Typology} 8(2). 149--211.
doi:\href{https://doi.org/10.1515/lity.2004.005}{10.1515/lity.2004.005}.

\leavevmode\vadjust pre{\hypertarget{ref-nubling2006}{}}%
Nübling, Damaris, Antje Dammel, Janet Duke \& Renata Szczepaniak. 2006.
\emph{Historische {Sprachwissenschaft} des {Deutschen}: Eine
{Einführung} in die {Prinzipien} des {Sprachwandel}s}. Tübingen: Narr.

\leavevmode\vadjust pre{\hypertarget{ref-olsen1981}{}}%
Olsen, Susan. 1981. \emph{Problems of {seem/scheinen} constructions and
their implications for the theory of predicate sentential
complementation} (Linguistische Arbeiten 96). Tübingen: Niemeyer.
doi:\href{https://doi.org/10.1515/9783111655833}{10.1515/9783111655833}.

\leavevmode\vadjust pre{\hypertarget{ref-pafel1989}{}}%
Pafel, Jürgen. 1989. \emph{Scheinen} + {Infinitiv}: Eine
oberflächengrammatische {Analyse}. In Gabriel Falkenberg (ed.),
\emph{{Wissen, Wahrnehmen, Glauben}: Epistemische {Ausdrücke} und
propositionale {Einstellungen}} (Linguistische Arbeiten 202), 123--172.
Tübingen: Niemeyer.
doi:\href{https://doi.org/10.18419/opus-8405}{10.18419/opus-8405}.

\leavevmode\vadjust pre{\hypertarget{ref-papemuller1980}{}}%
Pape-Müller, Sabine. 1980. \emph{Textfunktionen des passivs}
(Germanistische Linguistik 29). Tübingen: Niemeyer.
doi:\href{https://doi.org/10.1515/9783111370996}{10.1515/9783111370996}.

\leavevmode\vadjust pre{\hypertarget{ref-perlmutter1978}{}}%
Perlmutter, David M. 1978. Impersonal passives and the unaccusative
hypothesis. \emph{Proceedings of the Annual Meeting of the Berkeley
Linguistics Society} 4. 157--189.
\url{https://escholarship.org/uc/item/73h0s91v}.

\leavevmode\vadjust pre{\hypertarget{ref-pfeiffer1993}{}}%
Pfeiffer, Wolfgang. 1993. \emph{Etymologisches {Wörterbuch} des
{Deutschen}}. (Digitalised and revised version). Berlin: Digitales
{Wörterbuch} der deutschen {Sprache}.
\url{https://www.dwds.de/wb/wb-etymwb}.

\leavevmode\vadjust pre{\hypertarget{ref-pitteroff2014}{}}%
Pitteroff, Marcel. 2014. \emph{Non-canonical \emph{lassen}-middles}.
Universität Stuttgart PhD thesis.
doi:\href{https://doi.org/10.18419/opus-5396}{10.18419/opus-5396}.

\leavevmode\vadjust pre{\hypertarget{ref-plank2015}{}}%
Plank, Frans \& Aditi Lahiri. 2015. Macroscopic and microscopic
typology: Basic valence orientation, more pertinacious than meets the
naked eye. \emph{Linguistic Typology} 19(1). 1--54.
doi:\href{https://doi.org/10.1515/lingty-2015-0001}{10.1515/lingty-2015-0001}.

\leavevmode\vadjust pre{\hypertarget{ref-polenz1969}{}}%
Polenz, Peter von. 1969. Der {Pertinenzdativ} und seine {Satzbaupläne}.
In Ulrich Engel, Paul Grebe \& Heinz Rupp (eds.), \emph{Festschrift für
{Hugo Moser} zum 60. {Geburtstag}}, 146--171. Düsseldorf: Schwann.

\leavevmode\vadjust pre{\hypertarget{ref-primus2011}{}}%
Primus, Beatrice. 2011. Das unpersönliche {Passiv}: Ein {Fall} für die
{Konstruktionsgrammatik}? In Stefan Engelberg, Anke Holler \& Kristel
Proost (eds.), \emph{Sprachliches {Wissen} zwischen {Lexikon} und
{Grammatik}} (Jahrbuch {des} Instituts {für} Deutsche Sprache 2010),
285--314. Berlin: De Gruyter.
doi:\href{https://doi.org/10.1515/9783110262339.285}{10.1515/9783110262339.285}.

\leavevmode\vadjust pre{\hypertarget{ref-proost2009}{}}%
Proost, Kristel. 2009. Warum man nach {Schnäppchen} jagen, aber nicht
nach {Klamotten} bummeln kann: {Die} \emph{nach}-{Konstruktion} zwischen
{Lexikon} und {Grammatik}. In Edeltraud Winkler (ed.),
\emph{Konstruktionelle {Varianz} bei {Verben}} (Online {publizierte}
Arbeiten {zur} Linguistik 4/2009), 10--41. Mannheim: Institut für
Deutsche Sprache.
\url{https://nbn-resolving.org/urn:nbn:de:bsz:mh39-349}.

\leavevmode\vadjust pre{\hypertarget{ref-pullum1988}{}}%
Pullum, Geoffrey K. 1988. Citation etiquette beyond thunderdome.
\emph{Natural Language \& Linguistic Theory} 6(4). 579--588.
doi:\href{https://doi.org/10.1007/BF00134494}{10.1007/BF00134494}.

\leavevmode\vadjust pre{\hypertarget{ref-rapp1997}{}}%
Rapp, I. 1997. \emph{Partizipien und semantische {Struktur}: Zu
passivischen {Konstruktionen} mit dem 3. {Status}} (Studien {zur}
{deutschen} Grammatik 54). Tübingen: Stauffenburg.

\leavevmode\vadjust pre{\hypertarget{ref-rapp1996}{}}%
Rapp, Irene. 1996. Zustand? {Passiv}? {Überlegungen} zum sogenannten
{`{Zustandspassiv}'}. \emph{Zeitschrift für Sprachwissenschaft} 15(2).
231--265.
doi:\href{https://doi.org/10.1515/zfsw.1996.15.2.231}{10.1515/zfsw.1996.15.2.231}.

\leavevmode\vadjust pre{\hypertarget{ref-rapp2013}{}}%
Rapp, Irene \& Angelika Wöllstein. 2013. Satzwertige
\emph{zu}-{Infinitivkonstruktionen}. In Jörg Meibauer, Markus Steinbach
\& Hans Altmann (eds.), \emph{Satztypen des {Deutschen}}, 338--355.
Berlin: De Gruyter.
doi:\href{https://doi.org/10.1515/9783110224832.338}{10.1515/9783110224832.338}.

\leavevmode\vadjust pre{\hypertarget{ref-reis1976}{}}%
Reis, Marga. 1976. Zum grammatischen {Status} der {Hilfsverben}.
\emph{Beiträge zur Geschichte der deutschen Sprache und Literatur} 98.
64--82.
doi:\href{https://doi.org/10.1515/bgsl.1976.1976.98.64}{10.1515/bgsl.1976.1976.98.64}.

\leavevmode\vadjust pre{\hypertarget{ref-reis2005}{}}%
Reis, Marga. 2005. Zur {Grammatik} der sog. {`{Halbmodale}'}
\emph{drohen/versprechen} + {Infinitiv}. In Franz Josef D'Avis (ed.),
\emph{{Deutsche Syntax: Empirie und Theorie}} (Göteborger Germanistische
Forschungen 46). Göteborg: Acta Universitatis Gothoburgensis.
\url{http://hdl.handle.net/10900/47028}.

\leavevmode\vadjust pre{\hypertarget{ref-rothstein2007a}{}}%
Rothstein, Björn. 2007a. Die {Syntax} von {Fügungen} des {Typs}
\emph{kam gefahren}. \emph{Deutsche Sprache} 35(2). 159--172.
doi:\href{https://doi.org/10.37307/j.1868-775X.2007.02.05}{10.37307/j.1868-775X.2007.02.05}.

\leavevmode\vadjust pre{\hypertarget{ref-rothstein2007}{}}%
Rothstein, Björn. 2007b. Einige {Bemerkungen} zum {Partizip II} in
\emph{das {Pferd} hat die {Fesseln} bandagiert}. In Ljudmila Geist \&
Björn Rothstein (eds.), \emph{Kopulaverben und {Kopulasätze}.
{Intersprachliche} und intrasprachliche {Aspekte}} (Linguistische
Arbeiten 512), 285--298. Tübingen: Niemeyer.
doi:\href{https://doi.org/10.1515/9783110938838.285}{10.1515/9783110938838.285}.

\leavevmode\vadjust pre{\hypertarget{ref-rothstein2011}{}}%
Rothstein, Björn. 2011. Zur temporalen {Interpretation} von {Fügungen}
des {Typs} \emph{sie kamen gelaufen}. \emph{Zeitschrift für
germanistische Linguistik} 39(3). 356--376.
doi:\href{https://doi.org/10.1515/zgl.2011.027}{10.1515/zgl.2011.027}.

\leavevmode\vadjust pre{\hypertarget{ref-sapir1917}{}}%
Sapir, Edward. 1917. Review of {C.C. Uhlenbeck}: Het passieve karakter
van het verbum transitivum of van het verbum actionis in talen van
{Noord-Amerika}. \emph{International Journal of American Linguistics}
1(1). 82--86. \url{https://www.jstor.org/stable/1263405}.

\leavevmode\vadjust pre{\hypertarget{ref-sauerland1994}{}}%
Sauerland, Uli. 1994. German diathesis and verb morphology. In Douglas
A. Jones (ed.), \emph{Verb classes and alternations in {Bangla},
{German}, {English}, and {Korean}} (AI Memo 1517), 50--68. Cambridge,
MA: MIT. \url{http://hdl.handle.net/1721.1/7197}.

\leavevmode\vadjust pre{\hypertarget{ref-schafer2007}{}}%
Schäfer, Florian. 2007. \emph{On the nature of anticausative morphology:
External arguments in change-of-state contexts}. Universität Stuttgart
PhD thesis.
doi:\href{https://doi.org/10.18419/opus-5245}{10.18419/opus-5245}.

\leavevmode\vadjust pre{\hypertarget{ref-schallert2014}{}}%
Schallert, Oliver. 2014. \emph{Zur {Syntax} der
{Ersatzinfinitivkonstruktion}: {Typologie} und {Variation}} (Studien
{zur deutschen} Grammatik 87). Tübingen: Stauffenburg.

\leavevmode\vadjust pre{\hypertarget{ref-scheibl2006}{}}%
Scheibl, György. 2006. Aktiv, {Passiv} und {Antipassiv}. {Argumentale
Reorganisation} im {Deutschen}. \emph{Deutsche Sprache} 34(4). 354--382.
doi:\href{https://doi.org/10.37307/j.1868-775X.2006.04.05}{10.37307/j.1868-775X.2006.04.05}.

\leavevmode\vadjust pre{\hypertarget{ref-schlucker2007}{}}%
Schlücker, Barbara. 2007. \emph{Bleiben}: Eine unterspezifizierte
{Kopula}. In Ljudmila Geist \& Björn Rothstein (eds.),
\emph{Kopulaverben und {Kopulasätze}: {Intersprachliche} und
intrasprachliche {Aspekte}} (Linguistische Arbeiten 512), 141--164.
Tübingen: Niemeyer.
doi:\href{https://doi.org/10.1515/9783110938838.141}{10.1515/9783110938838.141}.

\leavevmode\vadjust pre{\hypertarget{ref-schmid2005}{}}%
Schmid, Tanja. 2005. \emph{Infinitival syntax: {Infinitivus Pro
Participio} as a repair strategy} (Linguistics Today 79). Amsterdam:
Benjamins. doi:\href{https://doi.org/10.1075/la.79}{10.1075/la.79}.

\leavevmode\vadjust pre{\hypertarget{ref-schumacher1986}{}}%
Schumacher, Helmut (ed.). 1986. \emph{Verben in {Feldern}:
{Valenzwörterbuch} zur {Syntax} und {Semantik} deutscher {Verben}}
({Schriften des Instituts für deutsche Sprache} 1). Berlin: De Gruyter.
doi:\href{https://doi.org/10.1515/9783110861853}{10.1515/9783110861853}.

\leavevmode\vadjust pre{\hypertarget{ref-schwarz2004}{}}%
Schwarz, Christian. 2004. \emph{Die \emph{tun}-{Periphrase} im
{Deutschen}}. LMU München Master's thesis.
\url{https://nbn-resolving.org/urn:nbn:de:bsz:25-opus-17597}.

\leavevmode\vadjust pre{\hypertarget{ref-seiler1973}{}}%
Seiler, Hansjakob. 1973. On the semanto-syntactic configuration
{`{Possessor} of an {Act}'}. In Braj B. Kachru (ed.), \emph{Issues in
linguistics: Papers in honor of {Henry and Renée Kahane}}. Urbana, Ill.:
University of Illinois Press.

\leavevmode\vadjust pre{\hypertarget{ref-silverstein1972}{}}%
Silverstein, Michael. 1972. Chinook jargon: Language contact and the
problem of multi-level generative systems, {I}. \emph{Language} 48(2).
378--406. \url{https://www.jstor.org/stable/412141}.

\leavevmode\vadjust pre{\hypertarget{ref-smirnova2016}{}}%
Smirnova, Elena. 2016. Die {Entwicklung} des deutschen
\emph{zu}-{Infinitivs}: Eine {Korpusstudie}. \emph{Beiträge zur
Geschichte der deutschen Sprache und Literatur} 138(4). 491--523.
doi:\href{https://doi.org/10.1515/bgsl-2016-0039}{10.1515/bgsl-2016-0039}.

\leavevmode\vadjust pre{\hypertarget{ref-speyer2018a}{}}%
Speyer, Augustin. 2018. The {ACI} construction in the history of
{German}. In Agnes Jäger, Gisella Ferraresi \& Helmut Weiß (eds.),
\emph{Clause structure and word order in the history of {German}}
(Oxford Studies in Diachronic and Historical Linguistics 28), 324--347.
Oxford: Oxford University Press.
doi:\href{https://doi.org/10.1093/oso/9780198813545.003.0017}{10.1093/oso/9780198813545.003.0017}.

\leavevmode\vadjust pre{\hypertarget{ref-stathi2010}{}}%
Stathi, Katerina. 2010. Is {German} \emph{gehören} an auxiliary? The
grammaticalization of the construction \emph{gehören} + participle {II}.
In Katerina Stathi, Elke Gehweiler \& Ekkehard König (eds.),
\emph{Grammaticalization: Current views and issues} (Studies in Language
Companion Series 119), 323--342. Amsterdam: Benjamins.
doi:\href{https://doi.org/10.1075/slcs.119.17sta}{10.1075/slcs.119.17sta}.

\leavevmode\vadjust pre{\hypertarget{ref-steinbach1998}{}}%
Steinbach, Markus. 1998. \emph{Middles in {German}}. Berlin:
Humboldt-Universität PhD thesis.
doi:\href{https://doi.org/10.18452/14603}{10.18452/14603}.

\leavevmode\vadjust pre{\hypertarget{ref-stiebels1996}{}}%
Stiebels, Barbara. 1996. \emph{Lexikalische {Argumente} und {Adjunkte}:
Zum semantischen {Beitrag} von verbalen {Präfixen} und {Partikeln}}
(Studia Grammatica 39). Berlin: Akademie Verlag.
doi:\href{https://doi.org/10.1515/9783050072319}{10.1515/9783050072319}.

\leavevmode\vadjust pre{\hypertarget{ref-stopp1957}{}}%
Stopp, Frederick John. 1957. \emph{A manual of modern {German}}. London:
University Tutorial Press.

\leavevmode\vadjust pre{\hypertarget{ref-storch1978}{}}%
Storch, Günther. 1978. \emph{Semantische {Untersuchungen} zu den
inchoativen {Verben} im {Deutschen}} (Schriften {zur} Linguistik 9).
Braunschweig: Vieweg.
doi:\href{https://doi.org/10.1007/978-3-322-86211-2}{10.1007/978-3-322-86211-2}.

\leavevmode\vadjust pre{\hypertarget{ref-stotzel1970}{}}%
Stötzel, Georg. 1970. \emph{Ausdruckseite und {Inhaltsseite} der
{Sprache}: Methodenkritische {Studien} am {Beispiel} der deutschen
{Reflexivverben}} (Linguistische Reihe 3). München: Hueber.

\leavevmode\vadjust pre{\hypertarget{ref-strecker2017}{}}%
Strecker, Bruno. 2017. Behelfe ich mir oder mich? {Kasus} des
{Reflexivums} bei \emph{behelfen}. \emph{Grammatik in {Fragen} und
{Antworten}}. Leibniz-Institut für Deutsche Sprache.
\url{https://grammis.ids-mannheim.de/fragen/4248}.

\leavevmode\vadjust pre{\hypertarget{ref-strobl2007}{}}%
Strobel, Sven. 2008. \emph{Die {Perfektauxiliarselektion} des
{Deutschen}: {Ein} lexikalistischer {Ansatz} ohne {Unakkusativität}}.
Universität Stuttgart PhD thesis.
doi:\href{https://doi.org/10.18419/opus-5257}{10.18419/opus-5257}.

\leavevmode\vadjust pre{\hypertarget{ref-szatmari2002}{}}%
Szatmári, Petra. 2002. \emph{Das gehört nicht vom {Tisch}
gewischt\ldots{}}: {Überlegungen} zu einem modalen {Passiv} und dessen
{Einordnung} ins {Passiv-Feld}. \emph{Jezikoslovlje} 3(1-2). 171--192.
\url{https://hrcak.srce.hr/31351}.

\leavevmode\vadjust pre{\hypertarget{ref-thieroff2007}{}}%
Thieroff, Rolf. 2007. \emph{Sein}: {Kopula}, {Passiv-} und/oder
{Tempus-Auxiliar}? In Ljudmila Geist \& Björn Rothstein (eds.),
\emph{Kopulaverben und {Kopulasätze}: {Intersprachliche} und
intrasprachliche {Aspekte}} (Linguistische Arbeiten 512), 165--180.
Tübingen: Niemeyer.
doi:\href{https://doi.org/10.1515/9783110938838.165}{10.1515/9783110938838.165}.

\leavevmode\vadjust pre{\hypertarget{ref-uhlig1883}{}}%
Uhlig, Gustavus. 1883. \emph{Dionysii thracis ars grammatici}
(Grammatici Graeci). Vol. 1. Leipzig: Teubner.

\leavevmode\vadjust pre{\hypertarget{ref-van-valin2004}{}}%
Van Valin, Robert Detrick Jr. 2004. Semantic macroroles in role and
reference grammar. In Rolf Kailuweit \& Martin Hummel (eds.),
\emph{Semantische {Rollen}}, 62--82. Tübingen: Narr.

\leavevmode\vadjust pre{\hypertarget{ref-vogel2007}{}}%
Vogel, Petra M. 2007. \emph{Anna ist essen!} {Neue} {Überlegungen} zum
{Absentiv}. In Ljudmila Geist \& Björn Rothstein (eds.),
\emph{Kopulaverben und {Kopulasätze}: {Intersprachliche} und
intrasprachliche {Aspekte}} (Linguistische Arbeiten 512), 253--284.
Berlin: De Gruyter.
doi:\href{https://doi.org/10.1515/9783110938838.253}{10.1515/9783110938838.253}.

\leavevmode\vadjust pre{\hypertarget{ref-weber2005}{}}%
Weber, Heinrich. 2005. Strukurverben im {Deutschen}. In Danuta
Stanulewicz, Roman Kalisz, Wilfried Kürschner \& Cäcilia Klaus (eds.),
\emph{De lingua et litteris: Studia in honorem {Casimiri Andreae
Sroka}}. Gdansk: Wydawnictwo Uniwersytetu Gdanskiego.
\url{http://hdl.handle.net/10900/46469}.

\leavevmode\vadjust pre{\hypertarget{ref-welke2011}{}}%
Welke, Klaus. 2011. \emph{Valenzgrammatik des {Deutschen}: Eine
{Einführung}} (De Gruyter Studium). Berlin: De Gruyter.
doi:\href{https://doi.org/10.1515/9783110254198}{10.1515/9783110254198}.

\leavevmode\vadjust pre{\hypertarget{ref-wiemer2007}{}}%
Wiemer, Björn \& Vladimir P. Nedjalkov. 2007. Reciprocal and reflexive
constructions in {German}. In Vladimir P. Nedjalkov (ed.),
\emph{Reciprocal constructions} (Typological Studies in Language 71),
455--512. Amsterdam: Benjamins.
doi:\href{https://doi.org/10.1075/tsl.71.17wie}{10.1075/tsl.71.17wie}.

\leavevmode\vadjust pre{\hypertarget{ref-wiese1996}{}}%
Wiese, Richard. 1996. \emph{The phonology of {German}} (The Phonology of
the World's Languages). Oxford: Oxford University Press.

\leavevmode\vadjust pre{\hypertarget{ref-wiskandt2022}{}}%
Wiskandt, Niklas. 2022. Paul ärgert sich, nervt sich aber nicht.
Semantische {Merkmale} deutscher {Objekt-Experiencer-Verben} und ihr
{Einfluss} auf {Antikausativkonstruktionen}. \emph{Germanistische
Werkstatt} 11. 245--259.
doi:\href{https://doi.org/10.25167/pg.4685}{10.25167/pg.4685}.

\leavevmode\vadjust pre{\hypertarget{ref-wunderlich1985}{}}%
Wunderlich, Dieter. 1985. Über die {Argumente} des {Verbs}.
\emph{Linguistische Berichte} 97. 183--227.

\leavevmode\vadjust pre{\hypertarget{ref-wunderlich1987}{}}%
Wunderlich, Dieter. 1987. An investigation of lexical composition: The
case of {German} \emph{be-} verbs. \emph{Linguistics} 25(2). 283--331.
doi:\href{https://doi.org/10.1515/ling.1987.25.2.283}{10.1515/ling.1987.25.2.283}.

\leavevmode\vadjust pre{\hypertarget{ref-wunderlich1993}{}}%
Wunderlich, Dieter. 1993. Diathesen. In Joachim Jacobs, Arnim von
Stechow, Wolfgang Sternefeld \& Theo Vennemann (eds.), \emph{Syntax: Ein
internationales {Handbuch} zeitgenössischer {Forschung}} (HSK 9/1),
730--747. Berlin: de Gruyter Mouton.
doi:\href{https://doi.org/10.1515/9783110095869.1.12.730}{10.1515/9783110095869.1.12.730}.

\leavevmode\vadjust pre{\hypertarget{ref-wunderlich1997}{}}%
Wunderlich, Dieter. 1997. Argument extension by lexical adjunction.
\emph{Journal of Semantics} 14. 95--142.
doi:\href{https://doi.org/10.1093/jos/14.2.95}{10.1093/jos/14.2.95}.

\leavevmode\vadjust pre{\hypertarget{ref-wunderlich2015}{}}%
Wunderlich, Dieter. 2015. Valency-changing word-formation. In Peter O.
Müller, Ingeborg Ohnheiser, Susan Olsen \& Franz Rainer (eds.),
\emph{Word-formation: An international handbook of the languages of
{Europe}} (HSK 40/2), 1424--1466. Berlin: de Gruyter Mouton.
doi:\href{https://doi.org/10.1515/9783110246278-039}{10.1515/9783110246278-039}.

\leavevmode\vadjust pre{\hypertarget{ref-wurmbrand2003}{}}%
Wurmbrand, Susanne. 2003. \emph{Infinitives: Restructuring and clause
structure} (Studies in Generative Grammar 55). Berlin: De Gruyter
Mouton.
doi:\href{https://doi.org/10.1515/9783110908329}{10.1515/9783110908329}.

\leavevmode\vadjust pre{\hypertarget{ref-ziegler2010}{}}%
Ziegler, Arne. 2010. \emph{Er erwartet sich nur das {Beste}}:
Reflexivierungstendenz und {Ausbau} des {Verbalparadigmas} in der
österreichischen {Standardsprache}. In Dagmar Bittner \& Gaeta Livio
(eds.), \emph{Kodierungstechniken im {Wandel}. Das {Zusammenspiel} von
{Analytik} und {Synthese} im {Gegenwartsdeutschen}}, 67--81. De Gruyter.
doi:\href{https://doi.org/10.1515/9783110228458.67}{10.1515/9783110228458.67}.

\leavevmode\vadjust pre{\hypertarget{ref-zifonun2003}{}}%
Zifonun, Gisela. 2003. \emph{Grammatik des {Deutschen} im europäischen
{Vergleich}: Das {Pronomen}. {Teil II}: {Reflexiv- und
Reziprokpronomen}} (Arbeitspapiere {und} Materialien {zur deutschen}
Sprache 1/03). Mannheim: Institut für Deutsche Sprache.
\url{https://nbn-resolving.org/urn:nbn:de:bsz:mh39-15840}.

\leavevmode\vadjust pre{\hypertarget{ref-zuniga2019}{}}%
Zúñiga, Fernando \& Seppo Kittilä. 2019. \emph{Grammatical voice}
(Cambridge Textbooks in Linguistics). Cambridge: Cambridge University
Press.
doi:\href{https://doi.org/10.1017/9781316671399}{10.1017/9781316671399}.

\end{CSLReferences}
