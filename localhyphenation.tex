%% hyphenation points for line breaks
%% Normally, automatic hyphenation in LaTeX is very good
%% If a word is mis-hyphenated, add it to this file
%%
%% add information to TeX file before \begin{document} with:
%% %% hyphenation points for line breaks
%% Normally, automatic hyphenation in LaTeX is very good
%% If a word is mis-hyphenated, add it to this file
%%
%% add information to TeX file before \begin{document} with:
%% %% hyphenation points for line breaks
%% Normally, automatic hyphenation in LaTeX is very good
%% If a word is mis-hyphenated, add it to this file
%%
%% add information to TeX file before \begin{document} with:
%% %% hyphenation points for line breaks
%% Normally, automatic hyphenation in LaTeX is very good
%% If a word is mis-hyphenated, add it to this file
%%
%% add information to TeX file before \begin{document} with:
%% \include{localhyphenation}
\hyphenation{
    ab-stumpf-en
    an-eig-nen
    Anti-kau-sa-tiv-kon-struk-tio-nen
    an-zu-schmei-cheln
    Auf-for-de-rungs-kau-sa-tiv
    auf-ge-stau-ten
    auf-stel-len
    auf-zieh-en
    ba-den
    be-ei-len
    Be-ga-bung
    be-kom-men
    Ber-lin
    be-schäf-ti-gen
    be-schwe-ren
    Bil-dungs-be-schrän-kung
    bre-chen
    bü-geln
    caus-a-tive
    cross-lin-guist-ic-al-ly
    Deut-sche
    dia-the-ses
    dia-the-sis
    durch-la-vie-ren
    du-schen
    ein-fres-sen
    ei-sen-be-schla-ge-nen
    ent-set-zen
    equi-pol-lent
    er-hei-tern
    er-wär-men
    Frei-tag
    Fun-die-rung
    Gar-ten
    ge-är-gert
    ge-ben
    ge-blie-ben
    ge-ring
    Ge-schich-te
    ge-schla-fen
    ge-schlos-sen
    her-ab-ge-stie-gen
    In-ter-sprach-li-che
    kal-ku-lie-ren
    klo-pfen
    kom-mu-ni-zie-ren
    kopf-rech-nen
    kra-chen
    kreu-zen
    mero-nym
    mero-nym-ic
    met-a-phor-ical
    mne-mo-nic
    mo-da-len
    move-ment
    nom-i-nal-iza-tion
    oblig-a-tory
    Opi-nia-tiv
    par-a-digm
    Par-ti-zip
    Po-li-zei-an-ga-ben
    pre-verb
    proof-read-ers
    quet-schen
    rein-kommt
    re-map-ping
    ren-tie-ren
    role-re-map-pings
    schät-zen
    schi-cken
    schleu-dern
    schlie-ßen
    schmei-ßen
    schön-ge-schwin-delt
    schrei-ben
    schwat-zen
    schwin-deln
    schwit-zen
    spa-ren
    stei-gen
    tank-en
    Tü-bin-gen
    über-ge-ben
    ver-kal-ku-lie-ren
    ver-kom-pli-zie-ren
    ver-krie-chen
    ver-mie-ten
    ver-schla-fen
    ver-spe-ku-lie-ren
    ver-ste-cken
    Vla-di-mir
    vor-sa-gen
    vor-schla-gen
    vor-stel-len
    Wahl-kampf
    warm
    wide-spread
    wie-der-ho-len
    win-ken
    Zeit-lu-pe
    zer-bre-chen
    zu-neh-men
    Zu-stands-pas-siv
}

\hyphenation{
    ab-stumpf-en
    an-eig-nen
    Anti-kau-sa-tiv-kon-struk-tio-nen
    an-zu-schmei-cheln
    Auf-for-de-rungs-kau-sa-tiv
    auf-ge-stau-ten
    auf-stel-len
    auf-zieh-en
    ba-den
    be-ei-len
    Be-ga-bung
    be-kom-men
    Ber-lin
    be-schäf-ti-gen
    be-schwe-ren
    Bil-dungs-be-schrän-kung
    bre-chen
    bü-geln
    caus-a-tive
    cross-lin-guist-ic-al-ly
    Deut-sche
    dia-the-ses
    dia-the-sis
    durch-la-vie-ren
    du-schen
    ein-fres-sen
    ei-sen-be-schla-ge-nen
    ent-set-zen
    equi-pol-lent
    er-hei-tern
    er-wär-men
    Frei-tag
    Fun-die-rung
    Gar-ten
    ge-är-gert
    ge-ben
    ge-blie-ben
    ge-ring
    Ge-schich-te
    ge-schla-fen
    ge-schlos-sen
    her-ab-ge-stie-gen
    In-ter-sprach-li-che
    kal-ku-lie-ren
    klo-pfen
    kom-mu-ni-zie-ren
    kopf-rech-nen
    kra-chen
    kreu-zen
    mero-nym
    mero-nym-ic
    met-a-phor-ical
    mne-mo-nic
    mo-da-len
    move-ment
    nom-i-nal-iza-tion
    oblig-a-tory
    Opi-nia-tiv
    par-a-digm
    Par-ti-zip
    Po-li-zei-an-ga-ben
    pre-verb
    proof-read-ers
    quet-schen
    rein-kommt
    re-map-ping
    ren-tie-ren
    role-re-map-pings
    schät-zen
    schi-cken
    schleu-dern
    schlie-ßen
    schmei-ßen
    schön-ge-schwin-delt
    schrei-ben
    schwat-zen
    schwin-deln
    schwit-zen
    spa-ren
    stei-gen
    tank-en
    Tü-bin-gen
    über-ge-ben
    ver-kal-ku-lie-ren
    ver-kom-pli-zie-ren
    ver-krie-chen
    ver-mie-ten
    ver-schla-fen
    ver-spe-ku-lie-ren
    ver-ste-cken
    Vla-di-mir
    vor-sa-gen
    vor-schla-gen
    vor-stel-len
    Wahl-kampf
    warm
    wide-spread
    wie-der-ho-len
    win-ken
    Zeit-lu-pe
    zer-bre-chen
    zu-neh-men
    Zu-stands-pas-siv
}

\hyphenation{
    ab-stumpf-en
    an-eig-nen
    Anti-kau-sa-tiv-kon-struk-tio-nen
    an-zu-schmei-cheln
    Auf-for-de-rungs-kau-sa-tiv
    auf-ge-stau-ten
    auf-stel-len
    auf-zieh-en
    ba-den
    be-ei-len
    Be-ga-bung
    be-kom-men
    Ber-lin
    be-schäf-ti-gen
    be-schwe-ren
    Bil-dungs-be-schrän-kung
    bre-chen
    bü-geln
    caus-a-tive
    cross-lin-guist-ic-al-ly
    Deut-sche
    dia-the-ses
    dia-the-sis
    durch-la-vie-ren
    du-schen
    ein-fres-sen
    ei-sen-be-schla-ge-nen
    ent-set-zen
    equi-pol-lent
    er-hei-tern
    er-wär-men
    Frei-tag
    Fun-die-rung
    Gar-ten
    ge-är-gert
    ge-ben
    ge-blie-ben
    ge-ring
    Ge-schich-te
    ge-schla-fen
    ge-schlos-sen
    her-ab-ge-stie-gen
    In-ter-sprach-li-che
    kal-ku-lie-ren
    klo-pfen
    kom-mu-ni-zie-ren
    kopf-rech-nen
    kra-chen
    kreu-zen
    mero-nym
    mero-nym-ic
    met-a-phor-ical
    mne-mo-nic
    mo-da-len
    move-ment
    nom-i-nal-iza-tion
    oblig-a-tory
    Opi-nia-tiv
    par-a-digm
    Par-ti-zip
    Po-li-zei-an-ga-ben
    pre-verb
    proof-read-ers
    quet-schen
    rein-kommt
    re-map-ping
    ren-tie-ren
    role-re-map-pings
    schät-zen
    schi-cken
    schleu-dern
    schlie-ßen
    schmei-ßen
    schön-ge-schwin-delt
    schrei-ben
    schwat-zen
    schwin-deln
    schwit-zen
    spa-ren
    stei-gen
    tank-en
    Tü-bin-gen
    über-ge-ben
    ver-kal-ku-lie-ren
    ver-kom-pli-zie-ren
    ver-krie-chen
    ver-mie-ten
    ver-schla-fen
    ver-spe-ku-lie-ren
    ver-ste-cken
    Vla-di-mir
    vor-sa-gen
    vor-schla-gen
    vor-stel-len
    Wahl-kampf
    warm
    wide-spread
    wie-der-ho-len
    win-ken
    Zeit-lu-pe
    zer-bre-chen
    zu-neh-men
    Zu-stands-pas-siv
}

\hyphenation{
    ab-stumpf-en
    an-eig-nen
    Anti-kau-sa-tiv-kon-struk-tio-nen
    an-zu-schmei-cheln
    Auf-for-de-rungs-kau-sa-tiv
    auf-ge-stau-ten
    auf-stel-len
    auf-zieh-en
    ba-den
    be-ei-len
    Be-ga-bung
    be-kom-men
    Ber-lin
    be-schäf-ti-gen
    be-schwe-ren
    Bil-dungs-be-schrän-kung
    bre-chen
    bü-geln
    caus-a-tive
    cross-lin-guist-ic-al-ly
    Deut-sche
    dia-the-ses
    dia-the-sis
    durch-la-vie-ren
    du-schen
    ein-fres-sen
    ei-sen-be-schla-ge-nen
    ent-set-zen
    equi-pol-lent
    er-hei-tern
    Er-leb-nis-kon-ver-siv
    er-wär-men
    Frei-tag
    Fun-die-rung
    Gar-ten
    ge-är-gert
    ge-ben
    ge-blie-ben
    ge-den-ken
    ge-ring
    Ge-schich-te
    ge-schla-fen
    ge-schlos-sen
    Häu-sern
    her-ab-ge-stie-gen
    In-ter-sprach-li-che
    kal-ku-lie-ren
    klein-krie-gen
    klo-pfen
    Kom-man-do
    kom-mu-ni-zie-ren
    kopf-rech-nen
    krab-beln
    kra-chen
    Krau-se
    kreu-zen
    mero-nym
    mero-nym-ic
    met-a-phor-ical
    mne-mo-nic
    mo-da-len
    mo-da-ler
    Mög-lich-keits-be-wer-tung
    move-ment
    müs-sen
    nom-i-nal-iza-tion
    oblig-a-tory
    Opi-nia-tiv
    par-a-digm
    Par-ti-zip
    Per-mis-siv-kau-sa-tiv
    Po-li-zei-an-ga-ben
    pre-verb
    proof-read-ers
    quet-schen
    rein-kommt
    Re-fle-xiv-pas-siv
    re-map-ping
    ren-tie-ren
    role-re-map-pings
    schät-zen
    schi-cken
    schleu-dern
    schlie-ßen
    schmei-ßen
    schön-ge-schwin-delt
    schrei-ben
    schwat-zen
    schwin-deln
    schwit-zen
    Smir-no-va
    spa-ren
    spe-cif-ic-ally
    stei-gen
    tank-en
    Tü-bin-gen
    über-ge-ben
    ver-kal-ku-lie-ren
    ver-kom-pli-zie-ren
    ver-krie-chen
    ver-mie-ten
    ver-schla-fen
    ver-spe-ku-lie-ren
    ver-stan-den
    ver-ste-cken
    Vla-di-mir
    vor-sa-gen
    vor-schla-gen
    vor-stel-len
    Wahl-kampf
    warm
    wide-spread
    wi-der-sprech-en
    wie-der-ho-len
    win-ken
    Zeit-lu-pe
    zer-bre-chen
    zu-neh-men
    Zu-stands-pas-siv
}
